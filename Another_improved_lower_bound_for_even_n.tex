\documentclass[11pt,letterpaper,reqno]{amsart}
\usepackage{tikz}
\usetikzlibrary{positioning, shapes.geometric, arrows.meta, calc, positioning}
\usepackage{chngcntr}
\usepackage{algorithm}
\usepackage{algpseudocode}
\usepackage{amsmath}
\usepackage{listings}
% \usepackage{xcolor}
\usepackage[dvipsnames]{xcolor}
\usepackage{mathtools}
\usetikzlibrary{calc, angles, quotes}
\usepackage{longtable}
\usepackage{mathtools}
\usepackage{breqn}


\lstdefinelanguage{Mathematica}{
  morekeywords={
    Module, If, Then, Else, Which, While, For, Do, Table, Return, Block, 
    Eigenvalues, AdjacencyMatrix, Select, Total, Max, Abs, Plot, ListPlot, 
    Import, Export, Solve, Simplify, Expand, Factor, Det, MatrixForm, NullSpace
  },
  sensitive=true,
  morecomment=[l]{(*},
  morestring=[b]",
}

\lstset{
  language=Mathematica,
  basicstyle=\ttfamily\small, 
  keywordstyle=\color{blue}, 
  commentstyle=\color{green!50!black}, 
  stringstyle=\color{orange!80!black}, 
  identifierstyle=\color{black}, 
  numberstyle=\tiny\color{gray},
  numbers=none,         
  breaklines=true,      
  breakatwhitespace=true,
  frame=single,         
  captionpos=b,         
  showstringspaces=false,
  columns=flexible,
  escapeinside={(*@}{@*)} 
}
\usepackage{amssymb}
\usepackage{amsmath}
\usepackage{amsthm}
\usepackage{amsfonts}
\usepackage{bbm}
\usepackage{enumitem} 
\usepackage{pgfplots}
\pgfplotsset{compat=1.18} 
\usepackage{booktabs}
\usepackage{graphicx}
\usepackage[T1]{fontenc}
\usepackage{doi}
\usepackage{float} 
\addtolength{\hoffset}{-1.5cm}\addtolength{\textwidth}{3cm}
\addtolength{\voffset}{-1cm}\addtolength{\textheight}{2cm}

\usepackage{bookmark}
\usepackage{hyperref}
\hypersetup{pdfstartview={FitH}}
\newcommand{\C}{\mathbb{C}}
\newcommand{\cE}{\mathcal{E}}
\newcommand{\norm}[1]{\lVert #1 \rVert}
\newcommand{\abs}[1]{| #1 |}
\newcommand{\bv}{\mathbf{v}}
\newcommand{\bw}{\mathbf{w}}
\newcommand{\tr}{\text{tr}}
\DeclareMathOperator{\rank}{rank}


\usepackage[capitalise]{cleveref}

\newtheorem{thm}{Theorem}[section]
\newtheorem{lem}[thm]{Lemma}
\newtheorem{prop}[thm]{Proposition}
\newtheorem{cor}[thm]{Corollary}
\newtheorem{claim}{Claim}
\newtheorem{ques}[thm]{Question}
\newtheorem{prob}[thm]{Problem}
\newtheorem{conj}[thm]{Conjecture}
\theoremstyle{definition}
\newtheorem{exm}[thm]{Example}
\newtheorem{remark}[thm]{Remark}
\newtheorem{target}[thm]{Target}
\newtheorem{defn}[thm]{Definition}
\numberwithin{equation}{section}
\newcommand{\N}{\mathbb{N}}
\newcommand{\taufunc}{\tau}
\newcommand{\omegap}{\omega}
\newcommand{\ord}{\operatorname{ord}}
\newcommand{\R}{\mathbb{R}}        % real numbers
\newcommand{\E}{\mathbb{E}}        % expectation
\newcommand{\Var}{\mathrm{Var}}    % variance
\newcommand{\Cov}{\operatorname{Cov}}
\newcommand{\PP}{\mathbb{P}}     % probability
\newcommand{\eps}{\varepsilon}     % epsilon
\newcommand{\ind}{\mathbf{1}}      % indicator function
\newcommand{\seq}[1]{\left(#1\right)} % sequence
\DeclareMathOperator{\Lip}{Lip}
\makeatother

\newcommand{\scc}[1]{\textcolor{red}{\textbf{[SC: } #1\textbf{]}}}
\newcommand{\adc}[1]{\textcolor{orange}{\textbf{[AD: } #1\textbf{]}}}
\newcommand{\yd}[1]{\textcolor{blue}{\textbf{[YD: } #1\textbf{]}}}
\newcommand{\qt}[1]{\textcolor{ForestGreen}{\textbf{[QT: } #1\textbf{]}}}

\begin{document}

\title{Another improved lower bound for even $n$}


% \author[S.~Cambie]{Stijn Cambie}
% \author[A.~Decadt]{Arne Decadt}
% \author[Y.~Dong]{Yanni Dong}
% \author[T.~Hu]{Tao Hu}
% \author[Q.~Tang]{Quanyu Tang}




% \address{}
% \email{}




% \begin{abstract}
% \end{abstract}

\maketitle
% \section{Introduction}
% On Bloom's Erd\H{o}s Problems website~\cite{TFB}, Problem~\#1045 asks the following:

% \begin{prob}\label{prob:init}
% Let $z_1,\dots,z_n\in\mathbb C$ with $|z_i-z_j|\le 2$ for all $i,j$, and
% \[
% \Delta(z_1,\dots,z_n)=\prod_{i\neq j}|z_i-z_j|.
% \]
% Maximise $\Delta$.
% \end{prob}


% \section{A uniform lower bound for even $n$}


For $z_1,\dots,z_n\in\mathbb C$ define
\[
\Delta(z_1,\dots,z_n):=\prod_{i\neq j}|z_i-z_j|.
\]
Let
\[
\Delta_n^*:=\sup\Bigl\{\Delta(z_1,\dots,z_n):\ \max_{i,j}|z_i-z_j|\le2\Bigr\}.
\]

\begin{thm}\label{thm:main}
Along even integers $n\to\infty$ one has
\[
\liminf_{\substack{n\to\infty\\ n\ \mathrm{even}}}\frac{\Delta_n^*}{n^n}
 \ge
\exp \left(\frac{7}{24}\zeta(3)-\frac{\pi^4}{864}\right)
 > 1.26852.
\]
In fact, there is an explicit family of configurations (with diameter~$2$) for which the ratio
$\Delta(z_1,\dots,z_n)/n^n$ converges to the constant on the right-hand side.
\end{thm}

\subsection{The triangular wave}
Let $\mathrm{tri}:\mathbb R\to\mathbb R$ be the $2\pi$-periodic function defined by
\[
\mathrm{tri}(x):=1-\frac{2}{\pi}\arccos{(\cos{x})}.
\]Equivalently, $\operatorname{tri}$ is the $2\pi$-periodic extension of the piecewise linear function on $[-\pi,\pi]$ given by
\[
\operatorname{tri}(x)=1-\frac{2}{\pi}|x| \qquad (x\in[-\pi,\pi]).
\]Define \(g(\theta):=\mathrm{tri}(3\theta)\).

\begin{lem}\label{lem:gprops}
The function $g$ satisfies:
\begin{enumerate}
\item $|g(\theta)|\le 1$ for all $\theta$.
\item $g$ is even: $g(-\theta)=g(\theta)$.
\item $g$ is $\pi$-anti\-periodic: $g(\theta+\pi)=-g(\theta)$.
\item $g$ is Lipschitz with constant $L:=6/\pi$, i.e.\ $|g(\theta)-g(\phi)|\le L|\theta-\phi|$.
\end{enumerate}
\end{lem}

\begin{proof}
Items (1) and (2) are immediate from the definition on $[-\pi,\pi]$.
For (3), since $\mathrm{tri}(x+\pi)=-\mathrm{tri}(x)$ and $3(\theta+\pi)=3\theta+3\pi\equiv 3\theta+\pi\pmod{2\pi}$,
we get $g(\theta+\pi)=\mathrm{tri}(3\theta+3\pi)=\mathrm{tri}(3\theta+\pi)=-\mathrm{tri}(3\theta)=-g(\theta)$.
For (4), $\mathrm{tri}$ has slope $\pm 2/\pi$ on each linear piece, hence is $(2/\pi)$-Lipschitz.
Composing with $3\theta$ multiplies the Lipschitz constant by~$3$, giving $L=6/\pi$.
\end{proof}

\subsection{The configuration for even $n$}
Fix an even integer $n=2m$.
Let
\[
\theta_k:=\frac{2\pi k}{n}\qquad (k=0,1,\dots,n-1),\qquad \zeta_k:=e^{i\theta_k}.
\]
Define the (small) parameter
\[
t_n:=\frac{\pi^2}{12n}\left(1-\frac{1}{n}\right),
\]
and the points
\[
z_k:=\bigl(1+t_n g(\theta_k)\bigr)\,\zeta_k,\qquad k=0,1,\dots,n-1.
\]

\subsection{Diameter bound: $|z_i-z_j|\le 2$ for large even $n$}

\begin{lem}\label{lem:diameter}
There exists $N_0$ such that for all even $n\ge N_0$ the configuration $\{z_k\}_{k=0}^{n-1}$ satisfies
\[
\max_{i,j}|z_i-z_j|\le 2.
\]
Moreover, for every $i$ we have $|z_i-z_{i+m}|=2$.
\end{lem}

\begin{proof}
Write $r_k:=1+t_n g(\theta_k)$, so $z_k=r_k e^{i\theta_k}$.

\medskip
\noindent\textbf{Antipodal pairs.}
Using $\zeta_{k+m}=-\zeta_k$ and Lemma~\ref{lem:gprops}(3), $g(\theta_{k+m})=-g(\theta_k)$, we get
\[
z_{k+m}=(1-t_n g(\theta_k))\,e^{i(\theta_k+\pi)}=-(1-t_n g(\theta_k))e^{i\theta_k},
\]
hence $z_k-z_{k+m}=2e^{i\theta_k}$ and therefore $|z_k-z_{k+m}|=2$.

\medskip
\noindent\textbf{General pairs.}
Fix $i\neq j$ and set $\beta:=|\theta_i-\theta_j|\in(0,\pi]$ (take the smaller arc).
Then
\[
|z_i-z_j|^2=r_i^2+r_j^2-2r_ir_j\cos\beta.
\]

\smallskip
\noindent\emph{Case 1: $\beta\le \pi/2$.}
Then $\cos\beta\ge0$, hence $|z_i-z_j|^2\le r_i^2+r_j^2\le 2(1+t_n)^2$.
Since $t_n\le \pi^2/24<\sqrt2-1$, we have $2(1+t_n)^2<4$, so $|z_i-z_j|<2$.





\smallskip
\noindent\emph{Case 2: $\beta\in(\pi/2,\pi)$.}
Write $\beta=\pi-\alpha$ where $\alpha\in(0,\pi/2)$.
Then $\cos\beta=-\cos\alpha$ and
\[
|z_i-z_j|^2=r_i^2+r_j^2+2r_ir_j\cos\alpha.
\]
We may write
\[
\theta_j \equiv \theta_i+\pi-\varepsilon\alpha\pmod{2\pi}
\quad\text{for some }\varepsilon\in\{\pm1\},
\]
with $\alpha=\tfrac{2\pi\ell}{n}$ for some integer $\ell\in\{1,2,\dots,\lfloor n/4\rfloor-1\}$.
By Lemma~\ref{lem:gprops}(3),
\[
g(\theta_j)=g(\theta_i+\pi-\varepsilon\alpha)=-g(\theta_i-\varepsilon\alpha).
\]
Set
\[
a:=g(\theta_i),\qquad b:=g(\theta_i-\varepsilon\alpha).
\]
Then $r_i=1+t_n a$ and $r_j=1-t_n b$.
A direct expansion gives
\begin{align}
|z_i-z_j|^2-4
&= -4\sin^2\!\frac{\alpha}{2}
+2t_n(a-b)(1+\cos\alpha)
+t_n^2\Bigl[(a-b)^2+2(1-\cos\alpha)\,ab\Bigr].
\label{eq:star}
\end{align}
We bound each term from above. By Lemma~\ref{lem:gprops}(4), $|a-b|\le L\alpha$ with $L=6/\pi$.
Also $|ab|\le1$ and $1-\cos\alpha\le \alpha^2/2$, hence
\[
(a-b)^2+2(1-\cos\alpha)ab\le (L\alpha)^2+\alpha^2=(L^2+1)\alpha^2.
\]
Moreover, for all $\alpha\ge0$ one has $\sin x\ge x-x^3/6$; applying this with $x=\alpha/2$ yields
$\sin^2(\alpha/2)\ge \alpha^2/4-\alpha^4/48$, hence
\[
-4\sin^2\frac{\alpha}{2}\le -\alpha^2+\frac{\alpha^4}{12}.
\]
Finally $1+\cos\alpha\le 2$, so $2t_n(a-b)(1+\cos\alpha)\le 4t_nL\alpha$.
Putting these bounds into \eqref{eq:star} gives
\begin{equation}\label{eq:hbound}
|z_i-z_j|^2-4
\le
-\alpha^2+\frac{\alpha^4}{12}+4t_nL\alpha+(L^2+1)t_n^2\alpha^2.
\end{equation}



Now substitute $\alpha=\frac{2\pi\ell}{n}$ and $t_n=\frac{\pi^2}{12n}(1-\frac1n)$. We now show that the right-hand side of~\eqref{eq:hbound} is $\le 0$ for all even $n\ge 8$. Since $t_n\le \pi^2/(12n)$ and $L^2+1=(\pi^2+36)/\pi^2$, we obtain
\[
4t_nL\alpha
=
4\cdot \frac{\pi^2}{12n}\Bigl(1-\frac1n\Bigr)\cdot \frac{6}{\pi}\cdot \frac{2\pi\ell}{n}
=
\frac{4\pi^2\ell}{n^2}\Bigl(1-\frac1n\Bigr),
\]
\[
\frac{\alpha^4}{12}
=
\frac{1}{12}\Bigl(\frac{2\pi\ell}{n}\Bigr)^4
=
\frac{4}{3}\frac{\pi^4\ell^4}{n^4},
\]
and
\[
(L^2+1)t_n^2\alpha^2
\le
\frac{\pi^2+36}{\pi^2}\cdot \frac{\pi^4}{144n^2}\cdot \Bigl(\frac{2\pi\ell}{n}\Bigr)^2
=
\frac{\pi^4(\pi^2+36)}{36}\cdot \frac{\ell^2}{n^4}.
\]
Moreover,
\[
-\alpha^2+4t_nL\alpha
=
-\Bigl(\frac{2\pi\ell}{n}\Bigr)^2+\frac{4\pi^2\ell}{n^2}\Bigl(1-\frac1n\Bigr)
=
-\frac{4\pi^2}{n^2}\Bigl(\ell(\ell-1)+\frac{\ell}{n}\Bigr).
\]
Therefore~\eqref{eq:hbound} implies the explicit bound
\begin{equation}\label{eq:E-nl}
|z_i-z_j|^2-4
\le
-\frac{4\pi^2}{n^2}\Bigl(\ell(\ell-1)+\frac{\ell}{n}\Bigr)
+\frac{\pi^4}{n^4}\Bigl(\frac{4}{3}\ell^4+\frac{\pi^2+36}{36}\ell^2\Bigr)
=:E_{n,\ell}.
\end{equation}

We will check $E_{n,\ell}\le 0$ by cases.

\smallskip
\noindent\emph{Case 1: $\ell=1$.}
Then \eqref{eq:E-nl} gives
\[
E_{n,1}
\le -\frac{4\pi^2}{n^3}
+\frac{\pi^4}{n^4}\Bigl(\frac{7}{3}+\frac{\pi^2}{36}\Bigr).
\]
Thus $E_{n,1}\le 0$ follows as soon as \(n \ge \frac{\pi^2}{4}\left(\frac{7}{3}+\frac{\pi^2}{36}\right)\approx 6.43372\). Hence $E_{n,1}\le 0$ for all $n\ge 8$.

\smallskip
\noindent\emph{Case 2: $\ell\ge 2$.}
Since $\ell(\ell-1)\ge \ell^2/2$ for $\ell\ge 2$, \eqref{eq:E-nl} yields
\[
E_{n,\ell}
\le
-\frac{2\pi^2\ell^2}{n^2}
+\frac{\pi^4}{n^4}\Bigl(\frac{4}{3}\ell^4+\frac{\pi^2+36}{36}\ell^2\Bigr)
=
\frac{\ell^2}{n^4}\Bigl(-2\pi^2 n^2+\pi^4\Bigl(\frac{4}{3}\ell^2+\frac{\pi^2+36}{36}\Bigr)\Bigr).
\]
Using $\ell\le n/4$ we have
\[
E_{n,\ell}
\le
\frac{\ell^2}{n^4}\Bigl(n^2\Bigl(-2\pi^2+\frac{\pi^4}{12}\Bigr)+\frac{\pi^4(\pi^2+36)}{36}\Bigr).
\]
For $n\ge 8$ we have
\[
n^2\Bigl(2-\frac{\pi^2}{12}\Bigr)\ \ge\ \frac{\pi^2(\pi^2+36)}{36},
\]so indeed $E_{n,\ell}\le 0$ for all $\ell\ge 2$ whenever $n\ge 8$.

\medskip
Combining the two cases, we have $E_{n,\ell}\le 0$ for every $1\le \ell\le n/4$ once $n\ge 8$.
Hence \eqref{eq:E-nl} implies $|z_i-z_j|^2-4\le 0$, i.e.\ $|z_i-z_j|\le 2$, for all near--antipodal pairs.
\end{proof}





\subsection{Factorization of $\Delta$ and a uniform bound on $\rho_{ij}$}



For $i\neq j$ define
\[
v_k:=g(\theta_k)\zeta_k,\qquad
\rho_{ij}:=\frac{v_i-v_j}{\zeta_i-\zeta_j}.
\]
Then
\[
z_i-z_j=(\zeta_i-\zeta_j)\bigl(1+t_n\rho_{ij}\bigr),
\]
and therefore
\begin{equation}\label{eq:factor}
\frac{\Delta(z_0,\dots,z_{n-1})}{\prod_{i\neq j}|\zeta_i-\zeta_j|}
=
\prod_{i\neq j}\bigl|1+t_n\rho_{ij}\bigr|.
\end{equation}

\begin{lem}\label{lem:rootsV}
For $\zeta_k=e^{2\pi i k/n}$ one has
\[
\prod_{i\neq j}|\zeta_i-\zeta_j|=n^n.
\]
\end{lem}

\begin{proof}
Let $p(z)=z^n-1=\prod_{j=0}^{n-1}(z-\zeta_j)$. For each root $\zeta_i$,
\[
p'(\zeta_i)=n\zeta_i^{n-1}=\prod_{j\neq i}(\zeta_i-\zeta_j).
\]
Taking absolute values and then the product over $i$ gives
\[
\prod_{i}\prod_{j\neq i}|\zeta_i-\zeta_j|
=\prod_i |p'(\zeta_i)|
=\prod_i n = n^n.
\qedhere\]
\end{proof}

Combining Lemma~\ref{lem:rootsV} with \eqref{eq:factor} yields
\[
\frac{\Delta(z_0,\dots,z_{n-1})}{n^n}
=
\prod_{i\neq j}\bigl|1+t_n\rho_{ij}\bigr|.
\]


\begin{lem}\label{lem:rhoBound}
There is a constant $M$ independent of $n$ such that $|\rho_{ij}|\le M$ for all $i\neq j$.
In fact one may take $M=4$.
\end{lem}

\begin{proof}
Write $g_i:=g(\theta_i)$. Then
\[
v_i-v_j=g_i(\zeta_i-\zeta_j)+(g_i-g_j)\zeta_j,
\]
hence
\[
\rho_{ij}=g_i+(g_i-g_j)\frac{\zeta_j}{\zeta_i-\zeta_j}.
\]
Therefore
\[
|\rho_{ij}|
\le |g_i|+\frac{|g_i-g_j|}{|\zeta_i-\zeta_j|}
\le 1+\frac{L|\theta_i-\theta_j|}{2|\sin((\theta_i-\theta_j)/2)|}.
\]
For $0<x\le\pi$, concavity of $\sin$ on $[0,\pi/2]$ gives $\sin(x/2)\ge \frac{x}{\pi}$, hence
\[
\frac{x}{2\sin(x/2)}\le \frac{\pi}{2}.
\]
Thus $|\rho_{ij}|\le 1+\frac{L\pi}{2}=1+\frac{6}{\pi}\cdot\frac{\pi}{2}=4$.
\end{proof}

\subsection{Second-order expansion of $\log(\Delta/n^n)$}
Taking logs,
\begin{equation}\label{eq:logsum}
\log\frac{\Delta(z_0,\dots,z_{n-1})}{n^n}
=\sum_{i\neq j}\log\bigl|1+t_n\rho_{ij}\bigr|.
\end{equation}

\begin{lem}\label{lem:taylor}
Assume $|u|\le 1/2$. Then
\[
\log|1+u|=\Re\!\left(u-\frac{u^2}{2}\right)+R(u),
\qquad |R(u)|\le \frac{2}{3}|u|^3.
\]
\end{lem}

\begin{proof}
For $|u|<1$ we have $\log(1+u)=\sum_{k\ge1}(-1)^{k+1}u^k/k$.
Taking real parts gives $\log|1+u|=\Re\log(1+u)$.
The tail satisfies
\[
\left|\sum_{k\ge3}\frac{(-1)^{k+1}}{k}u^k\right|
\le \sum_{k\ge3}\frac{|u|^k}{k}
\le \frac{1}{3}\sum_{k\ge3}|u|^k
=\frac{|u|^3}{3(1-|u|)}\le \frac{2}{3}|u|^3.
\qedhere\]
\end{proof}

\begin{lem}\label{lem:linearCancel}
Let $n=2m$ be even. Then $\sum_{i\neq j}\rho_{ij}=0$.
\end{lem}

\begin{proof}
We have $\zeta_{k+m}=-\zeta_k$.
By Lemma~\ref{lem:gprops}(3),
$g(\theta_{k+m})=-g(\theta_k)$, hence
\[
v_{k+m}=g(\theta_{k+m})\zeta_{k+m}=(-g(\theta_k))(-\zeta_k)=v_k.
\]
Thus for any $i\neq j$,
\[
\rho_{i+m,\;j+m}=\frac{v_{i+m}-v_{j+m}}{\zeta_{i+m}-\zeta_{j+m}}
=\frac{v_i-v_j}{-(\zeta_i-\zeta_j)}=-\rho_{ij}.
\]
The map $(i,j)\mapsto(i+m,j+m)$ is a bijection on ordered pairs $i\neq j$, so the sum cancels.
\end{proof}

\begin{lem}\label{lem:logAsym}
Along even $n\to\infty$,
\[
\log\frac{\Delta(z_0,\dots,z_{n-1})}{n^n}
=
-\frac{t_n^2}{2}\sum_{i\neq j}\Re(\rho_{ij}^2)+o(1).
\]
\end{lem}

\begin{proof}
For all sufficiently large $n$, Lemma~\ref{lem:rhoBound} and $t_n\le \pi^2/(12n)$ imply
$|t_n\rho_{ij}|\le 1/2$.
Apply Lemma~\ref{lem:taylor} to \eqref{eq:logsum} termwise:
\[
\log\frac{\Delta}{n^n}
=\sum_{i\neq j}\Re\!\left(t_n\rho_{ij}-\frac{t_n^2}{2}\rho_{ij}^2\right)
+\sum_{i\neq j}R(t_n\rho_{ij}).
\]
By Lemma~\ref{lem:linearCancel}, the linear term vanishes.
For the remainder, using $|R(u)|\le \frac23|u|^3$ and $|\rho_{ij}|\le M$,
\[
\left|\sum_{i\neq j}R(t_n\rho_{ij})\right|
\le \frac{2}{3}\,n(n-1)\,(t_n M)^3
=O(n^2 t_n^3)=O(1/n)\to0.
\qedhere\]
\end{proof}


\subsection{A Riemann-sum limit for $\sum\Re(\rho_{ij}^2)$}
Define for $x,y\in[0,2\pi]$,
\[
\xi(x):=e^{ix},\qquad f(x):=g(x)e^{ix},\qquad
\rho(x,y):=\frac{f(x)-f(y)}{\xi(x)-\xi(y)}\quad(x\neq y),
\]
and $F(x,y):=\Re(\rho(x,y)^2)$ for $x\neq y$, with $F(x,x):=0$.
Then $\rho(\theta_i,\theta_j)=\rho_{ij}$.

\begin{lem}\label{lem:Riemann}
The function $F$ is bounded on $[0,2\pi]^2$ and is continuous off the diagonal $\{x=y\}$.
Hence $F$ is Riemann integrable and
\[
\lim_{n\to\infty}\frac{1}{n^2}\sum_{i,j=0}^{n-1}F(\theta_i,\theta_j)
=
\frac{1}{4\pi^2}\int_0^{2\pi}\int_0^{2\pi}F(x,y)\,dx\,dy.
\]
Consequently,
\[
\lim_{n\to\infty}\frac{1}{n^2}\sum_{i\neq j}\Re(\rho_{ij}^2)
=
\frac{1}{4\pi^2}\int_0^{2\pi}\int_0^{2\pi}\Re(\rho(x,y)^2)\,dx\,dy
=:\,J.
\]
\end{lem}

\begin{proof}
Boundedness follows from the same estimate as in Lemma~\ref{lem:rhoBound} (with $\theta_i-\theta_j$ replaced by $x-y$).
Continuity holds for $x\neq y$ since the denominator is nonzero.
The diagonal has measure~$0$, hence $F$ is Riemann integrable.
For Riemann integrable functions, uniform-grid Riemann sums converge to the integral.
Finally, adding/removing the diagonal terms changes the sum by at most $O(n)$, hence $o(n^2)$ after normalization.
\end{proof}

\subsection{Computing $J$ via Fej\'er approximation and Fourier orthogonality}

\subsubsection{Fourier series of the triangular wave}
\begin{lem}\label{lem:triFourier}
The $2\pi$-periodic function $\mathrm{tri}$ has the Fourier expansion
\[
\mathrm{tri}(x)=\sum_{\substack{k\ge1\\ k\ \mathrm{odd}}}\frac{8}{\pi^2k^2}\cos(kx),
\]
with absolute (hence uniform) convergence.
Consequently,
\[
g(\theta)=\mathrm{tri}(3\theta)
=\sum_{\substack{r\ge1\\ r\ \mathrm{odd}}}\frac{8}{\pi^2r^2}\cos(3r\theta).
\]
\end{lem}

\begin{proof}
Since $\mathrm{tri}$ is even and has mean~$0$, only cosine coefficients appear:
\[
a_k=\frac{1}{\pi}\int_{-\pi}^{\pi}\mathrm{tri}(x)\cos(kx)\,dx
=\frac{2}{\pi}\int_0^{\pi}\left(1-\frac{2x}{\pi}\right)\cos(kx)\,dx.
\]
The first term integrates to $0$. For the second term, integration by parts gives
\[
\int_0^\pi x\cos(kx)\,dx=\frac{(-1)^k-1}{k^2},
\]
hence
\[
a_k=\frac{4}{\pi^2}\cdot\frac{1-(-1)^k}{k^2}
=
\begin{cases}
\frac{8}{\pi^2k^2},& k\ \text{odd},\\
0,& k\ \text{even}.
\end{cases}
\]
Absolute convergence follows from $\sum_{k\ \mathrm{odd}}1/k^2<\infty$.
\end{proof}

\subsubsection{A kernel orthogonality identity}
For integers $k$ define, for $x\neq y$,
\[
R_k(x,y):=\frac{e^{ikx}-e^{iky}}{e^{ix}-e^{iy}}.
\]

\begin{lem}\label{lem:RkIntegral}
For integers $k,\ell$ one has
\[
\frac{1}{4\pi^2}\int_0^{2\pi}\int_0^{2\pi} R_k(x,y)\,R_\ell(x,y)\,dx\,dy
=
\begin{cases}
1-|k-1|,& k+\ell=2,\\
0,& k+\ell\neq 2.
\end{cases}
\]
\end{lem}

\begin{proof}
Let $z=e^{ix}$ and $w=e^{iy}$.
If $k\ge1$ then
\[
R_k(x,y)=\frac{z^k-w^k}{z-w}=\sum_{m=0}^{k-1} z^{k-1-m}w^m
=\sum_{m=0}^{k-1} e^{i((k-1-m)x+my)}.
\]
If $k\le0$ then, writing $k=-p$ with $p\ge0$,
\[
R_{-p}(x,y)=\frac{z^{-p}-w^{-p}}{z-w}
=\frac{w^p-z^p}{z^p w^p(z-w)}
=-\frac{z^p-w^p}{z^p w^p(z-w)}
=-\sum_{m=0}^{p-1} z^{-(m+1)}w^{-(p-m)}.
\]
In all cases, $R_kR_\ell$ is a finite sum of exponentials $e^{i(ax+by)}$.
Using
\[
\frac{1}{2\pi}\int_0^{2\pi} e^{iax}\,dx=
\begin{cases}
1,& a=0,\\
0,& a\neq 0,
\end{cases}
\]
one checks that a nonzero contribution can occur only when the total $x$-frequency and $y$-frequency both vanish,
which forces $k+\ell=2$.
When $k+\ell=2$, the number of surviving terms is $1-|k-1|$, and each contributes $1$.
\end{proof}

\subsubsection{Fej\'er approximation to justify passage to the Fourier side}

Let $K_N$ be the \emph{Fej\'er kernel}
\[
K_N(t):=\frac{1}{N+1}\left(\frac{\sin((N+1)t/2)}{\sin(t/2)}\right)^2\ge0,
\qquad \frac{1}{2\pi}\int_0^{2\pi}K_N(t)\,dt=1.
\]
Define the \emph{Fej\'er mean} of a $2\pi$-periodic continuous $f$ by
\[
f^{(N)}(x):=\frac{1}{2\pi}\int_0^{2\pi} f(x-t)\,K_N(t)\,dt.
\]
Then $f^{(N)}$ is a trigonometric polynomial of degree $\le N$ and $f^{(N)}\to f$ uniformly as $N\to\infty$.

\begin{lem}\label{lem:FejerLip}
If $f$ is Lipschitz with constant $\Lip(f)$, then $\Lip(f^{(N)})\le \Lip(f)$ for all $N$.
Moreover, for all $x\neq y$,
\[
\left|\frac{f^{(N)}(x)-f^{(N)}(y)}{e^{ix}-e^{iy}}\right|
\le \frac{\pi}{2}\,\Lip(f),
\]
uniformly in $N$.
\end{lem}

\begin{proof}
Since $K_N$ is a probability kernel,
\begin{align*}
|f^{(N)}(x)-f^{(N)}(y)|
&=\left|\frac{1}{2\pi}\int_0^{2\pi}\bigl(f(x-t)-f(y-t)\bigr)K_N(t)\,dt\right|\\
&\le \frac{1}{2\pi}\int_0^{2\pi}\Lip(f)\,|x-y|\,K_N(t)\,dt
=\Lip(f)\,|x-y|.
\end{align*}
Thus $\Lip(f^{(N)})\le \Lip(f)$.
Also $|e^{ix}-e^{iy}|=2|\sin((x-y)/2)|$ and for $0<|x-y|\le \pi$ we have
$\sin(|x-y|/2)\ge |x-y|/\pi$, hence
$|x-y|/|e^{ix}-e^{iy}|\le \pi/2$.
\end{proof}

For our specific $f(x)=g(x)e^{ix}$, Lemma~\ref{lem:gprops} implies $|g|\le1$ and $\Lip(g)\le L$.
Hence
\[
|f(x)-f(y)|
\le |g(x)-g(y)|+|g(y)|\,|e^{ix}-e^{iy}|
\le L|x-y|+|x-y|=(L+1)|x-y|,
\]
so $\Lip(f)\le L+1$.

\begin{lem}\label{lem:JFourier}
Let $\rho(x,y)=\frac{f(x)-f(y)}{e^{ix}-e^{iy}}$ as above and define
\[
B(f):=\frac{1}{4\pi^2}\int_0^{2\pi}\int_0^{2\pi}\rho(x,y)^2\,dx\,dy.
\]
Then $B(f)$ is real and equals
\[
B(f)=\sum_{k\in\mathbb Z}\bigl(1-|k-1|\bigr)\,a_k\,a_{2-k},
\]
where $f(x)=\sum_{k\in\mathbb Z} a_k e^{ikx}$ is the Fourier series of $f$.
\end{lem}

\begin{proof}
Let $f^{(N)}$ be Fej\'er means. Define
\[
\rho_N(x,y):=\frac{f^{(N)}(x)-f^{(N)}(y)}{e^{ix}-e^{iy}}.
\]
By uniform convergence $f^{(N)}\to f$, we have $\rho_N(x,y)\to \rho(x,y)$ for every $x\neq y$.
By Lemma~\ref{lem:FejerLip}, $|\rho_N(x,y)|\le \frac{\pi}{2}\Lip(f)$ uniformly in $N$ and $(x,y)$.
Hence dominated convergence applies and
\[
B(f)=\lim_{N\to\infty}\frac{1}{4\pi^2}\iint \rho_N(x,y)^2\,dx\,dy.
\]
Now $f^{(N)}$ is a trigonometric polynomial, say
$f^{(N)}(x)=\sum_{|k|\le N} a_k^{(N)} e^{ikx}$.
Then for $x\neq y$,
\[
\rho_N(x,y)=\sum_{|k|\le N} a_k^{(N)}\,R_k(x,y),
\]
a finite sum. Therefore,
\[
\frac{1}{4\pi^2}\iint \rho_N(x,y)^2\,dx\,dy
=\sum_{|k|,|\ell|\le N} a_k^{(N)}a_\ell^{(N)}
\cdot \frac{1}{4\pi^2}\iint R_k R_\ell\,dx\,dy.
\]
By Lemma~\ref{lem:RkIntegral}, only terms with $k+\ell=2$ survive, giving
\[
\frac{1}{4\pi^2}\iint \rho_N(x,y)^2\,dx\,dy
=\sum_{k\in\mathbb Z}\bigl(1-|k-1|\bigr)a_k^{(N)}a_{2-k}^{(N)}.
\]
For Fej\'er means, $a_k^{(N)}=(1-|k|/(N+1))a_k$ for $|k|\le N$ and $0$ otherwise, hence $a_k^{(N)}\to a_k$
for each fixed $k$.
In our application the resulting infinite series is absolutely convergent (indeed it will reduce to $\sum_{r\ \mathrm{odd}}O(1/r^3)$),
so we may pass $N\to\infty$ termwise.
This yields the desired series for $B(f)$.
Since the series is real (coefficients are real in our case), $B(f)\in\mathbb R$.
\end{proof}

Finally, since $B(f)$ is real,
\[
J=\frac{1}{4\pi^2}\iint \Re(\rho^2)=\Re B(f)=B(f).
\]

\subsubsection{Evaluating the series for our $f$}
By Lemma~\ref{lem:triFourier},
\[
g(\theta)=\sum_{\substack{r\ge1\\ r\ \mathrm{odd}}}\frac{8}{\pi^2r^2}\cos(3r\theta).
\]
Hence
\[
f(\theta)=g(\theta)e^{i\theta}
=\sum_{\substack{r\ge1\\ r\ \mathrm{odd}}}\frac{4}{\pi^2r^2}\Bigl(e^{i(1+3r)\theta}+e^{i(1-3r)\theta}\Bigr).
\]
Therefore the Fourier coefficients of $f$ satisfy
\[
a_{1+3r}=a_{1-3r}=\frac{4}{\pi^2r^2}\quad (r\ge1,\ r\ \mathrm{odd}),
\qquad a_k=0\ \text{otherwise}.
\]
In Lemma~\ref{lem:JFourier}, only pairs $(k,2-k)$ contribute.
The only nonzero pairings are $k=1+3r$ with $2-k=1-3r$ (and vice versa), and for such $k$,
$1-|k-1|=1-3r$.
Thus
\begin{align*}
J
&=\sum_{r\ \mathrm{odd}\ge1} 2(1-3r)\left(\frac{4}{\pi^2r^2}\right)^2
=\frac{32}{\pi^4}\sum_{r\ \mathrm{odd}\ge1}\frac{1-3r}{r^4}.
\end{align*}
Using
\[
\sum_{r\ \mathrm{odd}}\frac1{r^4}=\left(1-\frac1{2^4}\right)\zeta(4)=\frac{15}{16}\cdot\frac{\pi^4}{90}=\frac{\pi^4}{96},
\qquad
\sum_{r\ \mathrm{odd}}\frac1{r^3}=\left(1-\frac1{2^3}\right)\zeta(3)=\frac{7}{8}\zeta(3),
\]
we obtain
\[
J=\frac{32}{\pi^4}\left(\frac{\pi^4}{96}-3\cdot\frac{7}{8}\zeta(3)\right)
=\frac{1}{3}-\frac{84\,\zeta(3)}{\pi^4}.
\]
In particular $J<0$.

\subsection{Asymptotic value of $\Delta/n^n$}
By Lemma~\ref{lem:logAsym} and Lemma~\ref{lem:Riemann},
\[
\log\frac{\Delta(z_0,\dots,z_{n-1})}{n^n}
=
-\frac{t_n^2}{2}\Bigl(n^2J+o(n^2)\Bigr)+o(1)
=
-\frac{(nt_n)^2}{2}J+o(1).
\]
Since $nt_n\to \pi^2/12$,
\[
\lim_{\substack{n\to\infty\\ n\ \mathrm{even}}}
\log\frac{\Delta(z_0,\dots,z_{n-1})}{n^n}
=
-\frac12\left(\frac{\pi^2}{12}\right)^2\left(\frac13-\frac{84\zeta(3)}{\pi^4}\right)
=
\frac{7}{24}\zeta(3)-\frac{\pi^4}{864}.
\]
Exponentiating,
\[
\lim_{\substack{n\to\infty\\ n\ \mathrm{even}}}
\frac{\Delta(z_0,\dots,z_{n-1})}{n^n}
=
\exp\left(\frac{7}{24}\zeta(3)-\frac{\pi^4}{864}\right).
\]By Lemma~\ref{lem:diameter}, for all sufficiently large even $n$ the configuration is feasible (diameter $\le2$),
hence $\Delta_n^*\ge \Delta(z_0,\dots,z_{n-1})$ and therefore
\[
\liminf_{\substack{n\to\infty\\ n\ \mathrm{even}}}\frac{\Delta_n^*}{n^n}\ge \exp\left(\frac{7}{24}\zeta(3)-\frac{\pi^4}{864}\right).
\]







% \begin{thebibliography}{99}

% \bibitem{TFB}
% T. F. Bloom, Erd\H{o}s Problem \#1045, \url{https://www.erdosproblems.com/1045}, accessed 2025-10-03

% % \bibitem{DIK-Annals}
% % P.~Deift, A.~Its, and I.~Krasovsky,
% % \newblock Asymptotics of Toeplitz, Hankel, and Toeplitz+Hankel determinants with Fisher–Hartwig singularities,
% % \newblock \emph{Annals of Mathematics} \textbf{174} (2011), 1243--1299.

% % \bibitem{DIK-MSRIP}
% % P.~Deift, A.~Its, and I.~Krasovsky,
% % \newblock On the asymptotics of a Toeplitz determinant with singularities,
% % \newblock in \emph{Recent Advances in Orthogonal Polynomials, Special Functions, and Their Applications}, MSRI Publications \textbf{65} (2014), 93--146; also arXiv:1206.1292.

% \bibitem{NocedalWright2006} J.~Nocedal and S.~J.~Wright, \textit{Numerical Optimization}, 2nd ed., Springer, New York, 2006.
% \end{thebibliography}














\end{document}

